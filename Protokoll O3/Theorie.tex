\section{Theoretische Grundlagen}

%\begin{wrapfigure}{r}{0.4\textwidth} % r=rechts, l=links, 0.4=Breite 
%        \centering
%        \includegraphics[width=0.9\linewidth]{Bilder/Titelbild.png}
%        \caption{Veranschaulichung der LSA}
%       \label{fig:SpASkizze}
%\end{wrapfigure}

  \subsection{Elektromagnetische Wellen}
Ausgehend von der Wellennatur des lässt sich eine Lichtwelle an Punkt P durch ihre elektrische Feldstärke beschreiben. Diese Beträgt:\\

\begin{center}
   ${\vec{E}=\vec{A}cos(\omega*t-\vec{k}\vec{x}-\delta)}$ %Kursiv machen
\end{center}
Das Elektrische-Feld der Welle setzt sich zusammen aus dem A*cos wobei das A die amplitude und der cos die Welle beschreibt. Im Argument des cosinus ist zu finden:\\
Zum einen das $\omega*t$, welches die auslekung der schwingung in Abhängigkeit mit der zeit beschreibt, dann $\vec{k}$ beschreibt den Wellenzahlvektor, $\vec{x}$ den Ort der Welle und $\delta$ die Phasenkonstante.

\subsection{Interferenz}
Damit Wellen Interferieren müssen einige Eigenschaften zutreffen:\\
Die Wellen müssen Koheränt sein, 
dass bedeuted sie zueinander eine zeitlich feste Phasenbeziehung haben. Damit zwei Wellen miteinander Interferieren dürfen sie auch nicht senkrecht zueinander stehen. 
Wenn diese Bedingungen erfüllt werden dann können sich die Wellen gegenseitig beinflussen oder auch Interferieren. 
Die stärkste Form der Interferenz tritt auf wenn die Wellen zueinander jeweils um ganzzahlige $pi$ verschoben sind. 
Bei ungeraden $\pi$ kommt es zu destruktiver Interferenz, bei geraden $pi$ kommt es zu konstruktiver Interferenz